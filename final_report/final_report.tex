\documentclass[]{article}
\usepackage[T1]{fontenc}
\usepackage{lmodern}
\usepackage{amssymb,amsmath}
\usepackage{ifxetex,ifluatex}
\usepackage{fixltx2e} % provides \textsubscript
% use upquote if available, for straight quotes in verbatim environments
\IfFileExists{upquote.sty}{\usepackage{upquote}}{}
\ifnum 0\ifxetex 1\fi\ifluatex 1\fi=0 % if pdftex
  \usepackage[utf8]{inputenc}
\else % if luatex or xelatex
  \ifxetex
    \usepackage{mathspec}
    \usepackage{xltxtra,xunicode}
  \else
    \usepackage{fontspec}
  \fi
  \defaultfontfeatures{Mapping=tex-text,Scale=MatchLowercase}
  \newcommand{\euro}{€}
\fi
% use microtype if available
\IfFileExists{microtype.sty}{\usepackage{microtype}}{}
\usepackage{longtable,booktabs}
\ifxetex
  \usepackage[setpagesize=false, % page size defined by xetex
              unicode=false, % unicode breaks when used with xetex
              xetex]{hyperref}
\else
  \usepackage[unicode=true]{hyperref}
\fi
\hypersetup{breaklinks=true,
            bookmarks=true,
            pdfauthor={},
            pdftitle={},
            colorlinks=true,
            citecolor=blue,
            urlcolor=blue,
            linkcolor=magenta,
            pdfborder={0 0 0}}
\urlstyle{same}  % don't use monospace font for urls
\setlength{\parindent}{0pt}
\setlength{\parskip}{6pt plus 2pt minus 1pt}
\setlength{\emergencystretch}{3em}  % prevent overfull lines
\setcounter{secnumdepth}{0}

\author{}
\date{}

\begin{document}

\section{Sugarcane Field Burning}\label{sugarcane-field-burning}

\subsection{Final Report Quinn Hart
2015-03-15}\label{final-report-quinn-hart-2015-03-15}

This proposal developed an automated method to determine how much
sugarcane harvesting was burned as compared to mechanical harvesting.
The basic idea is that sugarcane farm boundaries are input into the
system, and the amount of burning over those regions are calculated.
This proposal primarily uses the MODIS burned area product (MCD45A1) as
a simple method of determining the amount of burning occurring during
the harvesting.

Tasks for the project included: * Application and Development *
Automated Report Generation * 2013 and 2014 MODIS Burn Summary Maps *
Evaluation Support * Cloud-based Processing Evaluation

\subsection{Application and
  Development}\label{application-and-development}

We developed core processing for the sugarcane fields, and provided data
to CARB. From those input data, we developed a workflow, which processed
these data, and provided detailed reports for the sugarcane processes,
and summaries of the burning. We provided CARB with all the MODIS pixels
used in the image processing. CARB has reviewed and approved the
methodology and the reports generated in this application development.

We have provided the source code for the project, in an open repository
(https://github.com/qjhart/sugarcane). The code is executed primarily in
PostGIS, and open source spatial database; and Grass, an open source GIS
tool which is used to process the raster imagery. The repository
includes all the code needed to create and maintain the GRASS database,
and to process the requests and create the output KML summarizations.

Further information is available for the 
\href{Application\%20Development}{Application Development}

\subsection{Automated Report
Generation}\label{automated-report-generation}

We have delivered burn summary maps, allowing CARB will be able to
process incoming request in-house. However, we continued to support
automated summary generation to allow CARB the ability to compare their
results with UC Davis results. More detailed information regarding can
be found regarding the
\href{Automated\%20Report\%20Generation}{Automated Report Generation}.

\subsubsection{Deliverables}\label{deliverables}

\begin{itemize}
\itemsep1pt\parskip0pt\parsep0pt
\item
  CARB submitted a number of new sugarcane fields in a mutually
  agreeable format. UCD processed the fields and returned automatically
  generated reports in the form of KML files.
\item
  The service was maintained for the duration of the project.
\item
  The summaries included a table of estimates of acreage burned on a
  per-field basis.
\end{itemize}

\subsection{2013 and 2014 MODIS Burn Summary
Maps}\label{and-2014-modis-burn-summary-maps}

We provided Burned Area Summary map for years 2011,2012, 2013, and 2014,
for the regions originally proposed, as well as an extension to those
regions based on additional field processing requests. These maps
provide the basis for CARB to develop in-house processing of sugarcane
burn requests.

\subsubsection{Deliverables:}\label{deliverables-1}

\begin{itemize}
\itemsep1pt\parskip0pt\parsep0pt
\item
  We delivered a number of maps showing all the MODIS derived burned
  area pixels, in a vector-based format. Each polygon corresponds to a
  MODIS pixel, and includes a summary of the day(s) of year the burn
  occurred.
\end{itemize}

\subsection{Evaluation Support}\label{evaluation-support}

UC Davis supported a number of expanded evaluation of the generated
results. In particular, to support CARB's burned area claims, UC Davis
compared MODIS derived product to similar products derived from higher
resolution imagery, most notably Landsat 7 and Landsat 8. UC Davis
provided evaluation as requested.

\paragraph{Deliverables:}\label{deliverables-2}

UC Davis provided three extended evaluation support reports.

\begin{itemize}
\itemsep1pt\parskip0pt\parsep0pt
\item
  \href{Destilaria\%20Alcidia}{Destilaria Alcidia}
\item
  \href{Jalles\%20Machado}{Jalles Machado}
\item
  \href{Monta\%20Alegre}{Monta Alegre}
\end{itemize}

These were supplied to CARB for the purposes of validating CARB's
original Burn summary evaluations. They included higher resolution
imagery and value added products to support or moderate CARB's original
MODIS based assessment, and a description of the processes used

\subsection{Cloud-based Processing
Evaluation}\label{cloud-based-processing-evaluation}

UC Davis will investigate additional cloud-based processing environments
for alternative evaluation methods of burned area detection. The
advantage of using a cloud based solution is that it eliminates the
current requirement of locally processing satellite data, which is a
burden for continued efforts by CARB. Additionally, it provides a path
to updating the processing methodology for new satellite sourced data.
UC Davis will investigate Google's Earth Engine processing environment
as potential platform for new detection methodologies.

\subsection{Deliverables:}\label{deliverables-3}

\begin{itemize}
\itemsep1pt\parskip0pt\parsep0pt
\item
  UC Davis will investigate Earth Engine as a potential platform for
  sugar-cane burning detection.
\item
  UC Davis will include some higher resolution (eg. Landsat) imagery in
  it's investigations.
\item
  UC Davis will investigate potential scripting methods to automate the
  evaluation.
\end{itemize}

\section{Methodology Overview}\label{methodology-overview}

\begin{itemize}
\itemsep1pt\parskip0pt\parsep0pt
\item
  Get Field List
\item
  Verify the fields are correct
\item
  Get a map of what was burned
\item
  Determine which fields were burned
\item
  Summarize the total number of fields
\item
  Compare to reported values
\item
  Adjust appropriately
\end{itemize}

\subsection{MODIS MCD45A1 Product}\label{modis-mcd45a1-product}

The MODIS Burned area product (MCD45A1) describes the day that a
particular pixel was burned throughout the year. In addition to the burn
date, the product includes: a qualitative estimate of the confidence of
the pixel; how many passes were detected; a coarse classification of the
land cover type of pixel (including cloud cover); and an indication of
how long during each month clouds continuously obscured the surface. The
burned product determines fires by detecting rapid changes in bands most
affected by burning vegetation. The change compares with pixels are over
the course of about 2 weeks, so areas under clouds can eventually be
detected. Changes are thresholded at a certain level, and there is not
an estimate on the extent of the burning in the pixel.

500m x 500m Daily Overpass SWIR vs.~Red Compare temporally Monthly
Product Get a product of day(s) pixel burned

\section{Automated Report
Generation}\label{automated-report-generation-1}

UC Davis was asked to process 34 separate farm applications. The
majority of these in the São Paulo State of Brazil, but also other
states, and one request for Costa Rica. In total there were about
135,000 individual fields included in the processing. \emph{Table P.1}
includes an overview of the processing included.

\emph{Table P.1:} Processing Overview

\begin{longtable}[c]{@{}llll@{}}
\toprule\addlinespace
identifier & Country & State & Number Fields 
\\\addlinespace
\midrule\endhead
\toprule\addlinespace
adecoagro\_monte\_alegre & BRA & Minas Gerais & 265
\\\addlinespace
alta\_mogiana & BRA & São Paulo & 1389
\\\addlinespace
alto\_alegre\_junqueira & BRA & Paraná & 6116
\\\addlinespace
biosev\_lem & BRA & São Paulo & 2926
\\\addlinespace
biosev\_sel & BRA & São Paulo & 5061
\\\addlinespace
biosev\_umb & BRA & São Paulo & 2663
\\\addlinespace
biosev\_vro & BRA & Minas Gerais & 5
\\\addlinespace
biosev\_vro & BRA & São Paulo & 7558
\\\addlinespace
bunge\_frutal & BRA & Mato Grosso do Sul & 207
\\\addlinespace
bunge\_frutal & BRA & Minas Gerais & 1970
\\\addlinespace
bunge\_ouroeste & BRA & São Paulo & 5821
\\\addlinespace
cargill\_mosaico & BRA & Minas Gerais & 119
\\\addlinespace
cargill\_mosaico & BRA & São Paulo & 2968
\\\addlinespace
catsa\_costa\_rica & CRI & Guanacaste & 1467
\\\addlinespace
conquista\_do\_pontal & BRA & São Paulo & 241
\\\addlinespace
copersucar & BRA & Minas Gerais & 4804
\\\addlinespace
copersucar\_cerradao & BRA & Minas Gerais & 4146
\\\addlinespace
destilaria\_alcidia & BRA & São Paulo & 88
\\\addlinespace
itb & BRA & Goiás & 2958
\\\addlinespace
itt & BRA & Minas Gerais & 5348
\\\addlinespace
jalles\_machado & BRA & Goiás & 518
\\\addlinespace
meridiano & BRA & São Paulo & 3145
\\\addlinespace
noble\_meridiano & BRA & São Paulo & 3145
\\\addlinespace
noble\_potireendaba & BRA & São Paulo & 3524
\\\addlinespace
odebrecht\_alto\_taquari & BRA & Mato Grosso & 70
\\\addlinespace
parceria & BRA & São Paulo & 2141
\\\addlinespace
propia & BRA & São Paulo & 782
\\\addlinespace
raizen\_costapinto & BRA & São Paulo & 224
\\\addlinespace
renuka & BRA & São Paulo & 23848
\\\addlinespace
santa\_candida & BRA & São Paulo & 3907
\\\addlinespace
sao\_joao & BRA & São Paulo & 3201
\\\addlinespace
sao\_luiz & BRA & São Paulo & 9342
\\\addlinespace
sao\_martinho & BRA & São Paulo & 4597
\\\addlinespace
solazyme & BRA & Minas Gerais & 522
\\\addlinespace
solazyme & BRA & São Paulo & 8217
\\\addlinespace
usm & BRA & São Paulo & 4652
\\\addlinespace
usmshp & BRA & São Paulo & 4597
\\\addlinespace
vista\_alegre & BRA & Mato Grosso do Sul & 2321
\\\addlinespace
\bottomrule
\end{longtable}

Table P.2 shows the two countries, and the States involved in
processing. São Paulo showed most requests, but five other states had
some processing requests.

\emph{Table P.2:} Countries and States included in sugarcane field
processing.

\begin{longtable}[c]{@{}lll@{}}
\toprule\addlinespace
Country & State & Fields
\\\addlinespace
\midrule\endhead
BRA & Goiás & 3476
\\\addlinespace
BRA & Mato Grosso & 70
\\\addlinespace
BRA & Mato Grosso do Sul & 2528
\\\addlinespace
BRA & Minas Gerais & 17179
\\\addlinespace
BRA & Paraná & 6116
\\\addlinespace
BRA & São Paulo & 104037
\\\addlinespace
CRI & Guanacaste & 1467
\\\addlinespace
\bottomrule
\end{longtable}

These included 4 seperate MODIS tiles that were included in the
processing. The MODIS tile extents, country and State boundaries, and
boundaries of the individual processing requests are shown in Figures
P.1 and P.2.

{[}{[} processing/bra.png \textbar{} width=700px {]}{]} \textbar{} ---
\textbar{} \emph{Figure P.1:} Brazil Processing Overview \textbar{}

{[}{[} processing/cri.png \textbar{} width=700px {]}{]} \textbar{} ---
\textbar{} \emph{Figure P.2:} Costa Rica Processing Overview \textbar{}

\section{Jalles Machado}\label{jalles-machado}

UCD was asked to investigate the extent of particular sugarcane fields
for Jalles Machado, as the MODIS evaluation potentially suggested
burning outside of the designated fields.

Evaluation of the area apparently shows an uncontrolled fire to the
north of the fields of interest. The fire is estimated at 10-15 sq km,
through an area that is more hilly, with some native vegetation. That
fire occurred at approximately the same time as some of the fields
identified as burned also burned.

\subsection{Introduction}\label{introduction}

The following request was given to UCD.

\begin{quote}
With regards to the Jalles Machado evaluation, and the burn pixels
detected during August 2014 (see screenshot below), is it possible to
obtain historic imagery for the same pixels (pre-2013)? John Courtis is
interested in the history, because it appears that the Company may be
engaged in land clearing for cultivation / expansion. Else, it could be
sugarcane straw / litter burn at the edge of the fields. In any case,
the Company claims that it was an accidental event with the fire
spreading to their Cristalina farms for which a police complaint was
filed by a motorist travelling near the farms. We want to make sure
there is no history to their claim.
\end{quote}

\subsection{Landsat8 Investigation}\label{landsat8-investigation}

Landsat 8 imagery was used to investigate the area before and after the
detected burn. \emph{Figure 2.} Shows the fields of interest with a
false color Landsat image right before the burn.

\begin{longtable}[c]{@{}ll@{}}
\toprule\addlinespace
{[}{[} overview.png \textbar{} width=400px {]}{]} & {[}{[} over.png
\textbar{} width=400px {]}{]}
\\\addlinespace
\midrule\endhead
\emph{Figure 1.} MODIS burns & \emph{Figure 2.} Location of Fields
\\\addlinespace
\bottomrule
\end{longtable}

\emph{Figures 3. and 4.} Show the same image, without the superimposed
fields, for the time before and after the fire. This false color image
uses (SWIR) for red, (NIR) for green, and (RED) for blue. In this
configuration, vegetation appears green, soil brown, and burned area
dark and purplish. Examination of these images already give a good
indication of the fire that occurred and it's extent.

\begin{longtable}[c]{@{}ll@{}}
\toprule\addlinespace
{[}{[} pre.png \textbar{} width=400px {]}{]} & {[}{[} post.png
\textbar{} width=400px {]}{]}
\\\addlinespace
\midrule\endhead
\emph{Figure 3.} Prefire false Color image & \emph{Figure 4.} Postfire
false Color image
\\\addlinespace
\bottomrule
\end{longtable}

\emph{Figures 5. and 6.} Show the normalized burn index (NBI) for the
same region and the same times. NBI is a normalized relationship between
the SWIR and the NIR Landsat bands. This relationship shows areas that
are likely to have been burned. Typically, the index is composed to show
fire scars as dark.

\begin{longtable}[c]{@{}ll@{}}
\toprule\addlinespace
{[}{[} pre\_nbi.png \textbar{} width=400px {]}{]} & {[}{[} post\_nbi.png
\textbar{} width=400px {]}{]}
\\\addlinespace
\midrule\endhead
\emph{Figure 5.} Prefire NBI & \emph{Figure 6.} Postfire NBI
\\\addlinespace
\bottomrule
\end{longtable}

Finally, \emph{Figures 7. and 8.} show the difference between these NBI
values. The darker area for the fire is not an indication of a more
severe burn necessarily, as the area started out greener, with a higher
NBI.

\begin{longtable}[c]{@{}ll@{}}
\toprule\addlinespace
{[}{[} w\_barc.png \textbar{} width=400px {]}{]} & {[}{[}
w\_barc\_fields.png \textbar{} width=400px {]}{]}
\\\addlinespace
\midrule\endhead
\emph{Figure 7.} NBI Difference & \emph{Figure 8.} NBI Difference with
fields
\\\addlinespace
\bottomrule
\end{longtable}

\subsection{Adedcoagro Monta Alegre -
2014-09-02}\label{adedcoagro-monta-alegre---2014-09-02}

For some farm regions, the MODIS (MCD45A1) Based Burned area detection,
underpredicts the reported burning for the Aldecoagro Monte Alegre
farms, for 2013. We investigate a Landsat approach, and look to the
process of automating a Landsat based approach for this. For this
example, we look at all the Landsat 8 imagerty for the region (Landsat 8
commenced in Apr, but probably captures most of the burning for this
region). As a quick test, we will compare the Normalized Burn Ratio for
two dates, before and after the harvesting season. We calculate the
Delta NBR between the dates, and Threshold these into the classes as
defined above.

\begin{longtable}[c]{@{}lll@{}}
\toprule\addlinespace
Landsat 8-day TOA Composite & Cloudy & Use
\\\addlinespace
\midrule\endhead
Apr 23, 2013 - May 1, 2013 & No & Start
\\\addlinespace
May 9, 2013 - May 17, 2013 & No
\\\addlinespace
May 25, 2013 - Jun 2, 2013 & Very
\\\addlinespace
Jun 10, 2013 - Jun 18, 2013 & No
\\\addlinespace
Jun 26, 2013 - Jul 4, 2013 & Very
\\\addlinespace
Jul 12, 2013 - Jul 20, 2013 & No
\\\addlinespace
Jul 28, 2013 - Aug 5, 2013 & No
\\\addlinespace
Aug 13, 2013 - Aug 21, 2013 & Very
\\\addlinespace
Aug 29, 2013 - Sep 6, 2013 & No
\\\addlinespace
Sep 14, 2013 - Sep 22, 2013 & Yes
\\\addlinespace
Sep 30, 2013 - Oct 8, 2013 & Yes
\\\addlinespace
Oct 16, 2013 - Oct 24, 2013 & Partly
\\\addlinespace
Nov 1, 2013 - Nov 9, 2013 & Yes
\\\addlinespace
Nov 17, 2013 - Nov 25, 2013 & No & End
\\\addlinespace
Dec 3, 2013 - Dec 11, 2013 & Partly (Scattered)
\\\addlinespace
Dec 19, 2013 - Dec 27, 2013 & Yes
\\\addlinespace
Jan 1, 2014 - Jan 9, 2014 & Partly (Scattered)
\\\addlinespace
\bottomrule
\end{longtable}

The following images show the comparison of the before and after burned
regions for part of the Monta ALegre farms. This area was chosen as it
has the highest variation in the DeltaNBR, and shows fields that are
both being harvested and fields that are growing back in 2013.

\begin{longtable}[c]{@{}lll@{}}
\toprule\addlinespace
Overview and Delta NBR & 2013-04-23 & 2013-11-17
\\\addlinespace
\midrule\endhead
{[}{[} overview.png \textbar{} width=200px {]}{]} & {[}{[} true04.png
\textbar{} width=200px {]}{]} & {[}{[} true11.png \textbar{} width=200px
{]}{]}{[}{[} burned.png \textbar{} width=200px {]}{]}
\\\addlinespace
\bottomrule
\end{longtable}

A comparison with the MODIS MCD45A Burned area product does not show a
strong correlation with areas with High Delta NBR. This is unexpected,
as the MCD45A product also uses a comparison of NIR and SWIR bands as
it's method of discerning burned areas. The region above doens't contain
and MODIS burned pixels, the example below compares the MCD45A MODIS
burned pixels with the Delta NBR for close region. Only a small subset
of the High Delta NBR areas are included, and the MCD45A product also
identifies areas for which the delta NBR is low, (Eg in the lower left
of the image.)

\begin{longtable}[c]{@{}lll@{}}
\toprule\addlinespace
Fields & Burns & Composite
\\\addlinespace
\midrule\endhead
{[}{[} fusion.png \textbar{} width=200px {]}{]} & {[}{[}
burn\_mosaic.png \textbar{} width=200px {]}{]} & {[}{[}
burn\_mosaic\_fields.png \textbar{} width=200px {]}{]}
\\\addlinespace
\bottomrule
\end{longtable}

\subsection{Example from one Field}\label{example-from-one-field}

The before burning data is Jun 18th, and the Post date is July 4th.
Although we see a difference in the brightness of the fields post
harvest, the normalized versions do not show much difference. This seems
to indicate that we are not seeing the difference between a harvested
and burned field. If we jump to September 6th, we see the color of the
fields become similar. I think what are seeing in this case is maybe a
defoliant (or lack of water) in Jun prior to Harvest.

\begin{longtable}[c]{@{}llll@{}}
\toprule\addlinespace
Date & True Color (432) & False Color (752) & NBR
\\\addlinespace
\midrule\endhead
06-18 & {[}{[} true-06-18.png \textbar{} width=200px {]}{]} & {[}{[}
fc-06-18.png \textbar{} width=200px {]}{]} & {[}{[}nbr-06-18.png
\textbar{} width=200px {]}{]}
\\\addlinespace
07-04 & {[}{[} true-07-04.png \textbar{} width=200px {]}{]} & {[}{[}
fc-07-04.png \textbar{} width=200px {]}{]} & {[}{[}nbr-07-04.png
\textbar{} width=200px {]}{]}
\\\addlinespace
09-06 & {[}{[} true-09-06.png \textbar{} width=200px {]}{]} & {[}{[}
fc-09-06.png \textbar{} width=200px {]}{]} & {[}{[}nbr-09-06.png
\textbar{} width=200px {]}{]}
\\\addlinespace
\bottomrule
\end{longtable}

Destilaria Alcidia - January 23rd, 2014

The MODIS based Burned area detection algorithm is based on discovering
changes in the spectral characteristics of the underlying pixels tuned
to discover changes related to the addition of charcoal and ash, and the
removal of vegetation. The MODIS based dataset is used primarily for
it's ease of use in determining burned areas, as the time of the burn is
included in the results.

The MODIS based algorithm identified burning in late September and early
October, as shown in Table 1.

Farm Burn Julian Days Burn Dates Reported Rebuilding 110001 270 Sep 26

\begin{verbatim}
110007
267,275
Sep 23, 
September
110013
281
Oct 8


110025
267,268,272,275
Sep 23,24,29, Oct 2
July
110039
275,280,282,289
Oct 2, 6, 8, 15
Oct / November
\end{verbatim}

The investigation below looks primarily at fields 110007,110025 and
110039 The fields identified as being mis-characterized with respect to
plantation rebuilding. Field 110136 was shown to burn at an earlier
date, Mar 11 2012.

Landsat imagery, when looked at temporally, can provide similar insights
at a finer scale, but a coarser temporal resolution. The following set
of images describe the changes in the fields, and their classification.
The red pixels indicate the locations of the MODIS pixels that were used
as the determining locations of burning. The images are at coarser time
scales because of the most infrequent visitation of the Landsat
Satellite, as well a few cloudy days in the time period of interest. For
Landsat imagery, a false color image, using band 7 (SWIR) as red, can
indicate (with reddish or purplish colors) and increase in the
reflectance in the SWIR as related to the NIR (band 4) and red (Band 3).
This increase is a typical signal for added charcoal and ash. In a
similar way, the Burned Area Index (BAI) is a red and NIR index which
compares a surface to the spectral signature of charcoal. In BAI
imagery, high values indicate a strong correlation to charcoal.

True Color (RGB) False Color (743) Burned Area Index (BAI)

\begin{verbatim}
Figure showing the changes in RGB, 743, and BAI over the time of observed burning.  The rows indicate dates 2012-08-12, 2012-08-28, 2012-09-13, and 2012-10-31.  The images show an increase in the BAI, and changes in the false color image, 743, consistent with added  charcoal and ash.
\end{verbatim}

The imagery shows a typical pattern for the the fields of interest. High
greenness in August, defoliation in late August, possibly due to
harvesting, and then a later increase in BAI, in September and October.
This could be related to bare soil alone, or burning, or burning
followed by clearing of the land.

The farms 1110025 and 110007 show an interesting pattern for the Landsat
overflight of Sept 19th, 2012. This particular date captures the
transition from the non-burned to burned identification. The pattern
seems to start from the North of the fields in an irregular pattern.

\begin{verbatim}
RGB Image
Fa743 Image
  



BAI
Panchromatic


Figure showing a more detailed image for farms 110025 and 110007, for Sept 19, 2012.  The images indicate irregularly high BAI in the Northern part of the fields.  This anomaly is visible in the BAI, RGB, False color (743), and Panchromatic images.
\end{verbatim}

\section{Historical Changes in Sugarcane
burning}\label{historical-changes-in-sugarcane-burning}

Since the project has compiled sugarcane burning maps over the years
2011-2014, it is possible to investigate some of the changes in burning
practices over that time. We investigated the changes in burn practices
for a number of states in Brazil, as well as Costa Rica.

\subsection{Regional Changes}\label{regional-changes}

Table \emph{H.1} Shows the changes in the total number of burned pixels
for regions where sugarcane fields were processed. Note, that these do
not necessarily cover the entire regions, but only those that overlapped
the 5 MODIS tiles used for the overall processing steps. Still, they
offer good insight into the changes over the last 4 years. The values
are in terms of individual MODIS pixels, which are about 25 hectares in
size.

\emph{Table H.1} Changes in the number of burned pixels for 2011 through
2014.

\begin{verbatim}
   region       | 2011  |  2012  | 2013  |  2014  
\end{verbatim}

\begin{longtable}[c]{@{}lllll@{}}
\toprule\addlinespace
Goiás & 56497 & 98822 & 20091 & 102923
\\\addlinespace
CRI & 471 & 1701 & 1914 & 1944
\\\addlinespace
Mato Grosso & 94046 & 196296 & 71184 & 121060
\\\addlinespace
Mato Grosso do Sul & 23449 & 67111 & 26282 & 10469
\\\addlinespace
Minas Gerais & 49844 & 45858 & 10771 & 42152
\\\addlinespace
Paraná & 3522 & 9464 & 5522 & 3831
\\\addlinespace
São Paulo & 20144 & 10155 & 5961 & 21716
\\\addlinespace
\bottomrule
\end{longtable}

These changes can be seen in Figure \emph{H.1}. There are a few items to
note in these data. First, that while there had been a general decrease
in burning, 2014 showed an increase in burning throughout most regions,
particularly as compared to 2014.

\emph{Figure H.1} Changes in burning by year

{[}{[}historical/by\_year.png{]}{]}

\subsection{Timing of Burning}\label{timing-of-burning}

We also looked at the average timing of the burning practices. These did
not change dramatically from year to year. From region to region, Costa
Rica shows a different schedule than Brazil, Table \emph{H.2}. Figure
\emph{H.2} shows as an example the trends in the day of year (doy),
burning for São Paulo. These bell shaped curves are typical for most
regions, and could be used to determine possible out of season burns
that might be reconsidered in terms of the estimation of mechanization.

\emph{Table H.2} Average Day of Burning by region

\begin{verbatim}
   region       | 2011 | 2012 | 2013 | 2014 
\end{verbatim}

\begin{longtable}[c]{@{}lllll@{}}
\toprule\addlinespace
Costa Rica & 91 & 80 & 71 & 74
\\\addlinespace
Goiás & 241 & 250 & 240 & 239
\\\addlinespace
Mato Grosso & 235 & 238 & 231 & 226
\\\addlinespace
Mato Grosso do Sul & 244 & 229 & 240 & 209
\\\addlinespace
Minas Gerais & 248 & 261 & 244 & 254
\\\addlinespace
Paraná & 232 & 237 & 209 & 135
\\\addlinespace
São Paulo & 231 & 234 & 230 & 233
\\\addlinespace
\bottomrule
\end{longtable}

\emph{Figure H.2} Example changes in seasonal burning practices.

\begin{longtable}[c]{@{}ll@{}}
\toprule\addlinespace
{[}{[} historical/sao\_paulo.png {]}{]} & {[}{[}
historical/costa\_rica.png {]}{]}
\\\addlinespace
\midrule\endhead
São Paulo & Costa Rica
\\\addlinespace
\bottomrule
\end{longtable}

A table of all weekly burns for all years is included in the summary
spreadsheet.

\end{document}
